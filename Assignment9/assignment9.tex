\documentclass[a4paper]{article}
\usepackage[a4paper]{geometry}
\usepackage{amsmath}
\usepackage{amsfonts}
\usepackage{amssymb}
\begin{document}
	\begin{center}
		\textbf{\LARGE Assignment 9}\linebreak\linebreak
		{\large AVIK BANERJEE (3374885), SOUMYADEEP BHATTACHARJEE (3375428)}
	\end{center}
\section*{Exercise 9.1}
\begin{enumerate}
\item[a)] A \textbf{method} can be defined as an approach or tool to accomplish a particular task, whereas \textbf{methodology} is the strategy or reasoning behind using that method. In the perspective of research, a \textit{method} is a procedure for collection and analysis of data relevant to some hypothesis and \textit{methodology} is the broader strategy that allowed the use of that \textit{method}. \linebreak\linebreak
As an example, let us consider a company \textbf{A} that has initiated a project to improve its website. To do so, it has created a survey using Likert scales and text input fields to record users' views on the usability of various areas of its website. Anyone visiting the website will have the choice to fill up the survey form. Here, \textbf{A} employs an online form based \textit{\textbf{survey method}} to collect relevant data. The strategy that allows it to use this method is called \textbf{\textit{survey research}}. That is \textit{survey research methodology} encompasses a wide range of \textit{methods} \textbf{A} can use to collect data from its users, from pen-and-paper based forms to in-depth one-on-one interviews. Methodology defines the overall strategy to be used for collection and analysis of data and method states how to implement that strategy.
\item[b)] In the situation described in the question, a \textbf{formative} evaluation study needs to be undertaken. A formative study will continue alongside the development of the interface and collect user feedback on the usability and information content of the interface. A summative assessment cannot be used in this case, since a summative assessment is used to collect data regarding the success of a project after the project is complete. In this case, we need to gather information regarding user demand in order to build the interface, which demands a formative evaluation study. 
\item[c)] In the first situation, a \textbf{within-subjects} study design can be used since it will require less participants, reduce errors arising out of individual differences, and will provide better comparison among the website designs.
In case we wish to study how men and women use the three websites, we need to employ a combination of between-subjects and within-subjects study designs, that is, we need to divide the participants into two groups based on their gender and then let both the groups test all three websites. This will help elicit differences between male and female reactions to the three websites.
\item[d)] The description in the question does not suggest a good controlled experiment as it does not say anything about other factors that may affect the attack of pests on the crops, such as the type of crop used, types of manure used, characteristics of the soil and so on.\linebreak\linebreak
This experiment can be improved by ensuring the same crops are used on both the fields with the same soil characteristics, similar usage of manure, similar environmental conditions such as exposure to sunlight and others, and then applying pesticide to one field and testing the effects. Another approach may include using the same field, carefully dividing the crops into two groups, one of which is exposed to pesticides, while the other is not, and then testing the effects on the two groups.
\end{enumerate}
\section*{Exercise 9.2}
\begin{enumerate}
	\item[a)] The example given in the question is an example of false positive error if we consider the matching of fingerprints as the experiment and the output as whether the fingerprints have matched. Now, some of the fingerprints matched even though they were different, thus indicating a false positive output.
	\item[b)] In this case, if the mismatch is incorrect, then the labs that reported the mismatch encountered false negatives, that is they could not find a match even though they existed. However, if the mismatch is correct, the labs which reported positive matches encountered false positives, that is they found matches even though the fingerprints did not match.
	\item[c)] The severity of harm caused by a type I or type II error depends on the situation and the status quo (or null hypothesis) itself. This can be illustrated with the help of two examples:\linebreak\linebreak
Let us consider a medical test for a disease. A false positive of a Type I error may give a patient some anxiety, but this will lead to other testing procedures which will ultimately reveal the initial test was incorrect. In contrast, a false negative from a Type II error would give a patient the incorrect assurance that he or she does not have a disease when he or she in fact does. As a result of this incorrect information, the disease would not be treated. If doctors could choose between these two options, a false positive is more desirable than a false negative.\linebreak
On the other hand, let us consider a murder trial. The null hypothesis here is that the person is not guilty. A Type I error would occur if the person were found guilty of a murder that he or she did not commit, which would be a very serious outcome for the defendant. On the other hand, a Type II error would occur if the jury finds the person not guilty even though he or she committed the murder, which is a great outcome for the defendant but not for society as a whole. Here we see the value in a judicial system that seeks to minimize Type I errors.
\end{enumerate}
\section*{Exercise 9.3}
\begin{enumerate}
	\item[a)] 3) Detergent
	\item[b)] 4) Record measurements and observations
	\item[c)] 1) Hypothesis
	\item[d)] 2) Change one variable at a time
	\item[e)] \textbf{Control group}: In this case the control group consists of a group of people wearing no sunscreen.\linebreak
	\textbf{Independent variables}:\begin{itemize}
		\item Whether sunscreen has been applied or not
		\item Duration of exposure to the sun
		\item Intensity of sunshine
		\item Gender of the test subjects
	\end{itemize}
\textbf{Dependent variable}: Whether the test subject has experienced sunburn
\item[f)] Latin square is a method to randomize allocation of test conditions and participants in large experiments in order to eliminate the effect of blocking factors. However, a latin square requires at least 3 blocking factors to be generated. That is, the rows and columns consist of members of two blocking factors and the third factor is applied at the intersections of each row and column, as values of the latin square. As a result, with only two conditions to be tested, the latin square cannot be used. 
\end{enumerate}
\end{document}